\section{Analytical BG--Insulin Model (Simulator Plant)}
\label{sec:plant-model}
%
This work uses the nonlinear glucose--insulin dynamics implemented in the \texttt{simglucose} virtual patient (see \texttt{simglucose/patient/t1dpatient.py}). The model is an ODE system with meal intake and subcutaneous insulin infusion as exogenous inputs.

\subsection{States, inputs, and outputs}
Let the state vector be
\begin{equation}
x(t)=
\begin{bmatrix}
Q_{sto}^{s} & Q_{sto}^{l} & Q_{gut} & Q_{1} & Q_{2} & I_{p} & X & I_{1} & I_{2} & I_{l} & S_{1} & S_{2} & G_{sc}
\end{bmatrix}^{\!\top},
\end{equation}
where \(Q_{sto}^{s}\) and \(Q_{sto}^{l}\) are solid and liquid glucose in the stomach (mg), \(Q_{gut}\) is intestinal glucose (mg), \(Q_{1},Q_{2}\) are glucose masses in plasma and peripheral compartments (mg/kg), \(I_p\) is plasma insulin mass (pmol/kg), \(I_l\) is liver insulin (pmol/kg), \(X\) is insulin action on utilization, \(I_1,I_2\) are delayed insulin action states, \(S_1,S_2\) are subcutaneous insulin compartments (pmol/(kg)), and \(G_{sc}\) is subcutaneous glucose mass (mg/kg).

The input vector is
\begin{equation}
u(t)=
\begin{bmatrix}
D(t)\\
u_I(t)
\end{bmatrix},
\end{equation}
where \(D(t)\) is the carbohydrate ingestion rate in mg/min and \(u_I(t)\) is the insulin infusion rate in pmol/(kg$\cdot$min). If the pump command is provided in units of U/min, the simulator converts it using \(1\,\mathrm{U}=6000\,\mathrm{pmol}\) and patient body weight \(BW\) (kg) as
\begin{equation}
u_I(t)=\frac{6000}{BW}\,u(t)\qquad [\mathrm{pmol/(kg\cdot min)}].
\end{equation}

The plasma glucose concentration and the subcutaneous (observable) glucose concentration are
\begin{equation}
G_p(t)=\frac{Q_1(t)}{V_g},\qquad
G_{sub}(t)=\frac{G_{sc}(t)}{V_g},
\end{equation}
where \(V_g\) is the glucose distribution volume (dL/kg).

\subsection{Meal absorption (stomach--gut)}
Define the total stomach content \(Q_{sto}(t)=Q_{sto}^{s}(t)+Q_{sto}^{l}(t)\). The stomach and gut dynamics are
\begin{align}
\dot Q_{sto}^{s} &= -k_{\max}Q_{sto}^{s} + D,\\
\dot Q_{sto}^{l} &= k_{\max}Q_{sto}^{s} - k_{gut}Q_{sto}^{l},\\
\dot Q_{gut} &= k_{gut}Q_{sto}^{l} - k_{abs}Q_{gut}.
\end{align}
The gastric emptying rate \(k_{gut}\) is a nonlinear function of \(Q_{sto}\) and an effective stomach content \(\bar D\) (constructed internally by the simulator from recently ingested and remaining stomach content):
\begin{equation}
k_{gut}=k_{\min}+\frac{k_{\max}-k_{\min}}{2}\left(
\tanh\!\big(a(Q_{sto}-b\bar D)\big)-\tanh\!\big(c(Q_{sto}-d\bar D)\big)+2
\right),
\end{equation}
with
\begin{equation}
a=\frac{5}{2\bar D(1-b)},\qquad c=\frac{5}{2\bar D d}.
\end{equation}
The rate of appearance of glucose in plasma is
\begin{equation}
R_a=\frac{f\,k_{abs}}{BW}\,Q_{gut}.
\end{equation}

\subsection{Glucose kinetics (two compartments)}
Endogenous glucose production (EGP) is modeled as
\begin{equation}
EGP = k_{p1}-k_{p2}Q_1-k_{p3}I_2.
\end{equation}
Renal excretion is thresholded:
\begin{equation}
E = k_{e1}\,\max(Q_1-k_{e2},\,0),
\end{equation}
and non--insulin-dependent utilization is \(U_{ii}=F_{snc}\).

Insulin-dependent glucose utilization uses
\begin{equation}
V_m = V_{m0}+V_{mx}X,\qquad K_m=K_{m0},\qquad
U_{id} = \frac{V_m\,Q_2}{K_m+Q_2}.
\end{equation}
The glucose mass dynamics are
\begin{align}
\dot Q_1 &= \max(EGP,0)+R_a-U_{ii}-E-k_1Q_1+k_2Q_2,\\
\dot Q_2 &= -U_{id}+k_1Q_1-k_2Q_2.
\end{align}

\subsection{Insulin kinetics and action}
Plasma insulin mass \(I_p\) evolves as
\begin{equation}
\dot I_p=-(m_2+m_4)I_p+m_1I_l+k_{a1}S_1+k_{a2}S_2,\qquad
I=\frac{I_p}{V_i},
\end{equation}
where \(I\) is plasma insulin concentration (pmol/L) and \(V_i\) is insulin distribution volume.

Insulin action dynamics are
\begin{align}
\dot X &= -p_{2u}X+p_{2u}(I-I_b),\\
\dot I_1 &= -k_i(I_1-I),\\
\dot I_2 &= -k_i(I_2-I_1),
\end{align}
where \(I_b\) is basal insulin concentration.

The liver insulin compartment and subcutaneous insulin absorption are
\begin{align}
\dot I_l &= -(m_1+m_{30})I_l+m_2 I_p,\\
\dot S_1 &= u_I-(k_{a1}+k_d)S_1,\\
\dot S_2 &= k_d S_1-k_{a2}S_2.
\end{align}

\subsection{Subcutaneous glucose}
Subcutaneous glucose is modeled as a first-order lag from plasma glucose mass:
\begin{equation}
\dot G_{sc}=-k_{sc}G_{sc}+k_{sc}Q_1,\qquad
G_{sub}=\frac{G_{sc}}{V_g}.
\end{equation}

\subsection{Safety and robustness indices}
\noindent The \textit{safety} of the system is quantified using the Safety Performance Index (SPI), defined as the time-averaged magnitude of constraint violations:
\begin{equation}
\mathcal{SPI} = \frac{1}{T} \int_{0}^{T} \Big( \max\!\big(0,\, y(t) - y_{\max}\big) + \max\!\big(0,\, y_{\min} - y(t)\big) \Big)\, dt,
\end{equation}
where \(y(t)\) is the glucose concentration at time \(t\), and \(y_{\min}\) and \(y_{\max}\) are the lower and upper safe bounds, respectively. \textit{Lower} values of \(\mathcal{SPI}\) indicate fewer/smaller safety violations, with \(\mathcal{SPI}=0\) corresponding to perfect safety (no violations).

\noindent The system's \textit{robustness} to dynamic disturbances is characterized by the Robustness Index (RI), defined as a disturbance-normalized integral tracking error:
\begin{equation}
\mathcal{RI} = \frac{1}{\|d\|_2} \int_{0}^{T} \big|y(t) - y_{\mathrm{ref}}\big|\, dt,
\qquad
\|d\|_2 \triangleq \left(\int_{0}^{T} d(t)^2\, dt\right)^{\tfrac12},
\end{equation}
where \(d(t)\) denotes the disturbance profile (e.g., meal/glucose appearance or other exogenous inputs) and \(y_{\mathrm{ref}}\) is the reference glucose level. \textit{Lower} values of \(\mathcal{RI}\) indicate smaller deviation from the reference per unit disturbance energy, hence higher robustness. (When the same disturbance profile is used across controllers, \(\|d\|_2\) is constant and \(\mathcal{RI}\) reduces to the integral absolute error up to a constant scaling.)

